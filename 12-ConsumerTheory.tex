\chapter{消费理论}
\setlength{\parskip}{0.5\baselineskip}

\section{偏好}

有的人说:“经济学本质上是关于‘选择’的科学。”这句话放在本节再合适不过。数学是所有学科走向抽象的必经之路,为了让偏好成为可研究的对象,我们需要用一些数学工具。

\subsection{构建偏好关系}

假设消费者只消费 $X$ 和 $Y$ 两种商品,称 $\left(X_1,Y_1\right)$ 为一个\textbf{消费组合},它代表消费者对两种商品的消费量。在 $\left(X_1,Y_1\right)$ 和 $\left(X_2,Y_2\right)$ 两个消费组合中,消费者选择 $\left(X_1,Y_1\right)$ 而不是 $\left(X_2,Y_2\right)$,如果他是理性人,选择的目标是让自己的幸福感最大化,那么可以推断出比起 $\left(X_2,Y_2\right)$,消费者更偏好 $\left(X_1,Y_1\right)$。

偏好有“作比较”的含义,比较消费组合是消费者在对它们按照偏好程度排序。我们知道大小关系的符号是 $\ge$,\textbf{偏好关系}的符号则是 $\succsim$,不同符号突出了它们是不同的序关系。$\succsim$ 由 $\succ$ 和 $\sim$ 组成,分别指的\textbf{严格偏好}和\textbf{无差异},合起来 $\succsim$ 代表“至少一样好”。

假设小明很喜欢吃苹果和香蕉,有多少吃多少,那么一定有 $(\text{苹果},\text{香蕉})=\left(3,4\right)\succsim\left(2,1\right)$,因为前一个组合对两种水果的消费量都大于后一个组合。但我们会发现 $\left(3,4\right)$ 和 $\left(4,3\right)$ 这样两个组合我们无法判断小明会选择哪一个,选择的结果只有小明自己知道。说明偏好关系确实能代表消费者的喜好,但外人不总能判断出消费者的偏好\footnote{数学中把这样的序关系称作偏序关系,其实 $\succ$ 这样弯曲的大于号正是偏序关系的符号}。所以我们要引入接下来的内容,把研究中心放在我们能判断的商品组合中。
