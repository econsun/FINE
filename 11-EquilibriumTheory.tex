\chapter{均衡理论}
\setlength{\parskip}{0.5\baselineskip}

对于经济学,图像是重要的;对于图像,坐标轴的变量是重要的,同时各种变量是经济学的研究目标,这就要求我们在看图的时候不能忽视“坐标轴代表的变量是什么”。从第一章开始我们就会接触图像,每学习一种曲线,我们就要本能地去关注它代表的是哪两个对应变量构成的轨迹,这样的学习方法可以帮助我们深刻地理解经济学曲线。

\section{需求}

\begin{definition}{需求量}
    在其他因素不变条件下,消费者在一定价格水平下愿意且能够支付的该商品数量。
\end{definition}

需求量的定义中体现了某个价格水平与某个商品数量的对应关系。这里“价格水平”是一个巧妙的表达,我们不仅可以研究某款特定电脑的在确定价格时消费者的需求量,也可以研究整个电脑行业的价格水平对应的消费者需求量,体现了使用“价格水平”表达的重要性。


\begin{definition}{需求}
    在其他因素不变条件下,消费者在一定时期内对于各种可能价格水平愿意且能够支付的该商品数量。
\end{definition}

在需求量的基础上,现在考虑多种价格水平。当得知消费者在各种价格水平下的需求量,便可以知道消费者对商品的需求。在这句话中就出现了“需求量”三个字,说明需求是由需求量构成的。既然我们掌握了价格水平与需求量的对应关系,这不正是在章首提到的“曲线是两个对应变量构成的轨迹”,就能画出需求曲线了。需求曲线上的每一个点都是消费者的需求量。



影响需求量的因素:1. 商品自身价格;2. 消费者收入水平;3. 相关商品价格;4. 消费者偏好;5. 消费者对未来预期;6. 消费者人数;7. 政策
