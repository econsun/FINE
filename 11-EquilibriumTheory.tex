\chapter{均衡理论}
\setlength{\parskip}{0.5\baselineskip}

对于经济学,图像是重要的;对于图像,坐标轴的变量是重要的,同时各种变量是经济学的研究目标,这就要求我们在看图的时候不能忽视“坐标轴代表的变量是什么”。从第一章开始我们就会接触图像,学习一种曲线时,我们要本能地去关注它代表的是哪两个对应变量构成的轨迹,这样的学习方法可以帮助我们深刻地理解经济学曲线。

\section{需求}

\subsection{需求量}

\begin{definition}{需求量}
    在其他因素不变时,消费者在一定价格水平下愿意且能够支付的该商品数量。
\end{definition}

需求量的定义中体现了单个价格水平与单个商品数量的对应关系。这里“价格水平”是一个巧妙的表达,我们不仅可以研究某款特定电脑的在确定价格时消费者的需求量,也可以研究整个电脑行业的价格水平(比如取所有电脑价格的平均数作为价格水平)对应的消费者需求量,体现了使用价格水平来表达的重要性。

\subsection{需求}

\begin{definition}{需求}
    在其他因素不变时,消费者在一定时期内对于各种可能价格水平愿意且能够支付的该商品数量。
\end{definition}

在需求量的基础上,我们得到需求的定义,开始考虑多种价格水平。根据定义,当得知消费者在各种价格水平下的需求量,便可以知道消费者对商品的需求。在这句话中就出现了“需求量”三个字,说明需求是由需求量构成的。既然我们掌握了价格水平与需求量的对应关系,这不正是在章首提到的“曲线是两个对应变量构成的轨迹”,就能画出需求曲线了。需求曲线上的每一个点都是消费者的需求量。

对比需求量和需求的定义,不难发现如“包子变贵后,我的需求变小了”是不准确的表达,因为与特定价格对应的是定义是需求量。虽然在表达中严格区分需求和需求量是有助于读者理解的做法,但也会增加作者的工作量,所以往后多数语义明确的场合,本书将混淆需求和需求量,统称为“需求”。

\subsection{需求函数}

\begin{definition}{需求函数}
    在其他因素不变时,简化版的需求函数为 $Q^d=f\left(P\right)$;更为复杂的马歇尔需求函数为 $Q_d=f\left(P_X,P_Y,I\right)$。
\end{definition}

马歇尔一派认为除了本商品价格会决定消费者的需求量,其他商品的价格和消费者的收入也会左右需求量,这是符合直觉的。

\subsection{影响需求量的因素}

影响需求量的因素有以下几种
\begin{enumerate}
    \item 自身商品价格(需求函数的自变量)
    \item 相关商品价格(需求函数的自变量)
    \item 消费者收入水平(需求函数的自变量)
    \item 消费者的预期
    \item 消费者的偏好
    \item 政策风向
    \item ……
\end{enumerate}

需求函数的自变量是需求量的决定性变量,毫无疑问会影响需求量。图 1-1 中,我们忽略其他商品价格和收入的影响,把本商品价格和对应的需求量画出来,得到结论:价格下降会伴随需求量下降,这是“点移动”。如果价格以外的因素导致需求量变化,即任意价格水平不变时需求量发生变化,则会发生“线移动”。有这样的区别,是因为价格是需求函数的内生变量,其他因素是需求函数的外生变量,即使马歇尔需求函数认为其他商品价格和收入也是需求的决定性因素,但在这里被我们抛弃了,也成为了影响需求曲线移动的外生变量。

\begin{figure}[H]
    \centering
    \begin{tikzpicture}
        \draw[thick] (0,5) node[left]{$\rm P$} -- (0,0) -- (5.3,0) node[below]{$\rm Q$};
        \draw[thick] (0.5,4.7) to [out=280,in=170](4.7,1) node[right]{$Q^d$};
    \end{tikzpicture}
    \caption{需求函数图像}
\end{figure}

\begin{experience}{有关作图}
    经经济学中的作图与数学中的有较大不同。一个被公认的马歇尔坐标系被允许不画箭头和原点,自变量在纵轴,因变量在横轴。如果你画出箭头和原点是无可厚非的,没人在乎这么多,但坐标轴绝不能画反。画图是经济学学生的基本功,应勤加练习,熟练掌握课本中的图像绘制。
\end{experience}