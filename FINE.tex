\immediate\write18{makeindex \jobname.idx}

\documentclass[a4paper,11pt]{book}
\usepackage[bindingoffset=1cm,top=2cm,bottom=2cm,left=2cm,right=2cm]{geometry}
\usepackage[super,square,comma,sort&compress]{natbib}

% 设置中文
\usepackage{xeCJK}
\setCJKmainfont{SimSun}[BoldFont=SimHei,ItalicFont=AR PL KaitiM GB]
\usepackage{CJKnumb}
\usepackage{zhnumber}

% 设置英文
\usepackage[T1]{fontenc}
\usepackage{anyfontsize}

% 基本设置
\usepackage{indentfirst}
\setlength\parindent{2em}
\usepackage{setspace}
\lineskip = 5pt
\lineskiplimit = 5pt

% 排版宏包
\usepackage[clearempty]{titlesec}
\usepackage{titletoc,titleps}
\usepackage{ifthen}
\usepackage{makeidx}
\makeindex

% 设置 enumerate
\usepackage{enumitem}
\setenumerate[1]{label=\textendash,itemsep=5pt,parsep=0\parskip,topsep=0pt}
\setitemize[1]{itemsep=5pt,partopsep=0pt,parsep=\parskip,topsep=5pt}
\setdescription{itemsep=5pt,partopsep=0pt,parsep=\parskip,topsep=5pt}

% 图表宏包
\usepackage{graphicx}
\usepackage{subfig}
\usepackage{caption}
\usepackage{bicaption}
\usepackage{array}
\usepackage{booktabs}
\usepackage{threeparttable}
\usepackage{makecell}
\usepackage{multirow}
\usepackage{multicol}
\usepackage{float}
\usepackage{paracol}
\usepackage{balance}
\usepackage[most]{tcolorbox}

\definecolor{browna}{rgb}{0.76,0.72,0.65}
\newtcbtheorem[no counter]{definition}{定义}{
    enhanced,
    sharp corners,
    attach boxed title to top left={
        yshifttext=-1mm
    },
    colback=white,
    colframe=browna,
    fonttitle=\bfseries,
    boxed title style={
        sharp corners,
        size=small,
        colback=browna,
        colframe=browna,
    } 
}{thm}

\definecolor{brownb}{rgb}{0.39,0.77,0.82}
\newtcbtheorem[no counter]{ramble}{漫谈}{
    enhanced,
    sharp corners,
    attach boxed title to top left={
        yshifttext=-1mm
    },
    colback=white,
    colframe=brownb,
    fonttitle=\bfseries,
    boxed title style={
        sharp corners,
        size=small,
        colback=brownb,
        colframe=brownb,
    } 
}{thm}

% 数学宏包
\usepackage{amsmath,bm}
\usepackage{amssymb}
\usepackage{esint}
\usepackage{amscd}
\usepackage{amsfonts}
\usepackage{mathrsfs}
\usepackage{oplotsymbl}
\usepackage{mathtools}
\usepackage{siunitx}
\usepackage{upgreek}
\usepackage{ntheorem}
\usepackage{empheq}

% 绘图宏包
\usepackage{animate}
\usepackage{tikz,pgf}
\usetikzlibrary{arrows.meta}
\usetikzlibrary{shadings}
\usetikzlibrary{backgrounds}
\usetikzlibrary{shapes.geometric}
\usetikzlibrary{shapes.symbols}
\usetikzlibrary{shapes.arrows}
\usetikzlibrary{shapes.multipart}
\usetikzlibrary{calc}
\usetikzlibrary{math}
\usetikzlibrary{angles}
\usetikzlibrary{quotes} %angles加文字注释
\usetikzlibrary{positioning}
\usetikzlibrary{decorations.pathmorphing,decorations.markings,decorations.shapes}
\usetikzlibrary{patterns,patterns.meta}
\usetikzlibrary{3d}
\usetikzlibrary{perspective}
\usetikzlibrary{through}
\usetikzlibrary{intersections}
\usetikzlibrary{matrix}
\usetikzlibrary{mindmap}
\usetikzlibrary{circuits.ee,circuits.ee.IEC,circuits.logic.IEC}
\usepackage{tikz-3dplot}
\usepackage{pgfplots}

% 超链接宏包
\usepackage[unicode,bookmarks,bookmarksnumbered,colorlinks,linkcolor=blue]{hyperref}%超链接

% 设置顶部对齐
\raggedbottom

% 开始文档
\begin{document}
\begin{spacing}{1.5}

% 设置图表公式编号格式
% \captionsetup[figure]{name=图, labelsep=space}
\renewcommand{\thefigure}{\arabic{chapter}-\arabic{figure}}
% \captionsetup[table]{name=表, labelsep=space}
\renewcommand{\thetable}{\arabic{chapter}-\arabic{table}}
\renewcommand{\theequation}{\arabic{chapter}-\arabic{equation}}

% 改为中文样式
\renewcommand{\bibname}{参考文献}
\renewcommand{\figurename}{图}
\renewcommand{\tablename}{表}
\renewcommand{\chaptername}{第\CJKnumber{\thechapter}章}

% 设置前言页眉
\newpagestyle{mypre}{
\sethead[\chaptertitle][][]
{}{}{\chaptertitle}
\setfoot[][\thepage][]{}{\thepage}{}
}

% 设置正文页眉页脚
\newpagestyle{main}{
\sethead[\thesection~\sectiontitle][][FINE]
{FINE}{}{\thesection~\sectiontitle}
\headrule
\setfoot[][\thepage][]{}{\thepage}{}
}

\newpagestyle{back}{
\sethead[\chaptertitle][][FINE]
{FINE}{}{\chaptertitle}
\headrule
\setfoot[][\thepage][]{}{\thepage}{}
}


% 压缩空白页
\setboolean{@twoside}{false}

% 开始前言
\frontmatter

\newcommand{\tm}{\fontspec{Times New Roman}}
\title{\huge\tm A Fantastic Intermediate Note on Economics\\\huge{(\,FINE, Wish Your Life Be Fine\,)}}
\author{\LARGE{\tm Paul Sun}}
\date{\LARGE\tm \today}
\maketitle

% 定义前页部分章起始格式
\titleformat{\chapter}[hang]{\Huge\bfseries\filcenter}{\chaptername}{1em}{}
\titleformat*{\section}{\huge\bfseries\filleft}
\titleformat*{\subsection}{\Large\bfseries}

% 修改前言和目录的页面样式
\thispagestyle{empty}
\input{0-preface}

% 从前言开始计算页码
\pagenumbering{Roman}
\setcounter{page}{0}

% 添加前页格式
\pagestyle{mypre}

\renewcommand{\contentsname}{目录\label{here}}	
\tableofcontents
\titlecontents{chapter}[0em]{}{\bf}{}{\titlerule*[0.5pc]{.}\contentspage[\pageref{here}]}
\addcontentsline{toc}{chapter}{目录}

\titlecontents{chapter}[1.5em]{}{\color{blue}\bf\contentslabel{1.5em}}{}{\titlerule*[0.5pc]{.}\contentspage[\pageref{here}]}

% 开始正文
\mainmatter
\setboolean{@twoside}{true}

\pagestyle{main}

% 增加段间距
% \setlength{\parskip}{0.5\baselineskip}

\chapter{均衡理论}
\setlength{\parskip}{0.5\baselineskip}

对于经济学,图像是重要的;对于图像,坐标轴的变量是重要的,同时各种变量是经济学的研究目标,这就要求我们在看图的时候不能忽视“坐标轴代表的变量是什么”。从第一章开始我们就会接触图像,每学习一种曲线,我们就要本能地去关注它代表的是哪两个对应变量构成的轨迹,这样的学习方法可以帮助我们深刻地理解经济学曲线。

\section{需求}

\begin{definition}{需求量}
    在其他因素不变条件下,消费者在一定价格水平下愿意且能够支付的该商品数量。
\end{definition}

需求量的定义中体现了某个价格水平与某个商品数量的对应关系。这里“价格水平”是一个巧妙的表达,我们不仅可以研究某款特定电脑的在确定价格时消费者的需求量,也可以研究整个电脑行业的价格水平对应的消费者需求量,体现了使用“价格水平”表达的重要性。


\begin{definition}{需求}
    在其他因素不变条件下,消费者在一定时期内对于各种可能价格水平愿意且能够支付的该商品数量。
\end{definition}

在需求量的基础上,现在考虑多种价格水平。根据定义,当得知消费者在各种价格水平下的需求量,便可以知道消费者对商品的需求。在这句话中就出现了“需求量”三个字,说明需求是由需求量构成的。既然我们掌握了价格水平与需求量的对应关系,这不正是在章首提到的“曲线是两个对应变量构成的轨迹”,就能画出需求曲线了。需求曲线上的每一个点都是消费者的需求量。



影响需求量的因素:1. 商品自身价格;2. 消费者收入水平;3. 相关商品价格;4. 消费者偏好;5. 消费者对未来预期;6. 消费者人数;7. 政策

\documentclass[../Main.tex]{subfiles}

\begin{document}

\chapter{消费理论}

\intro{我觉得噼里啪啦}

\section{偏好}

\defn{消费者选择理论}{你们好你们好你们好}

有的人说:“经济学本质上是关于‘选择’的科学。”这句话放在本节再合适不过。数学是所有学科走向抽象的必经之路,为了让偏好成为可研究的对象,我们需要用一些数学工具。

\subsection{构建偏好关系}

wajfioajifowjiofjaiowjfiowa

假设消费者只消费 $X$ 和 $Y$ 两种商品,称 $\left(X_1,Y_1\right)$ 为一个\textbf{消费组合},它代表消费者对两种商品的消费量。在 $\left(X_1,Y_1\right)$ 和 $\left(X_2,Y_2\right)$ 两个消费组合中,消费者选择 $\left(X_1,Y_1\right)$ 而不是 $\left(X_2,Y_2\right)$,如果他是理性人,选择的目标是让自己的幸福感最大化,那么可以推断出比起 $\left(X_2,Y_2\right)$,消费者更偏好 $\left(X_1,Y_1\right)$。

假设消费者只消费 $X$ 和 $Y$ 两种商品,称 $\left(X_1,Y_1\right)$ 为一个\textbf{消费组合},它代表消费者对两种商品的消费量。在 $\left(X_1,Y_1\right)$ 和 $\left(X_2,Y_2\right)$ 两个消费组合中,消费者选择 $\left(X_1,Y_1\right)$ 而不是 $\left(X_2,Y_2\right)$,如果他是理性人,选择的目标是让自己的幸福感最大化,那么可以推断出比起 $\left(X_2,Y_2\right)$,消费者更偏好 $\left(X_1,Y_1\right)$。

假设消费者只消费 $X$ 和 $Y$ 两种商品,称 $\left(X_1,Y_1\right)$ 为一个\textbf{消费组合},它代表消费者对两种商品的消费量。在 $\left(X_1,Y_1\right)$ 和 $\left(X_2,Y_2\right)$ 两个消费组合中,消费者选择 $\left(X_1,Y_1\right)$ 而不是 $\left(X_2,Y_2\right)$,如果他是理性人,选择的目标是让自己的幸福感最大化,那么可以推断出比起 $\left(X_2,Y_2\right)$,消费者更偏好 $\left(X_1,Y_1\right)$。

假设消费者只消费 $X$ 和 $Y$ 两种商品,称 $\left(X_1,Y_1\right)$ 为一个\textbf{消费组合},它代表消费者对两种商品的消费量。在 $\left(X_1,Y_1\right)$ 和 $\left(X_2,Y_2\right)$ 两个消费组合中,消费者选择 $\left(X_1,Y_1\right)$ 而不是 $\left(X_2,Y_2\right)$,如果他是理性人,选择的目标是让自己的幸福感最大化,那么可以推断出比起 $\left(X_2,Y_2\right)$,消费者更偏好 $\left(X_1,Y_1\right)$。

假设消费者只消费 $X$ 和 $Y$ 两种商品,称 $\left(X_1,Y_1\right)$ 为一个\textbf{消费组合},它代表消费者对两种商品的消费量。在 $\left(X_1,Y_1\right)$ 和 $\left(X_2,Y_2\right)$ 两个消费组合中,消费者选择 $\left(X_1,Y_1\right)$ 而不是 $\left(X_2,Y_2\right)$,如果他是理性人,选择的目标是让自己的幸福感最大化,那么可以推断出比起 $\left(X_2,Y_2\right)$,消费者更偏好 $\left(X_1,Y_1\right)$。

假设消费者只消费 $X$ 和 $Y$ 两种商品,称 $\left(X_1,Y_1\right)$ 为一个\textbf{消费组合},它代表消费者对两种商品的消费量。在 $\left(X_1,Y_1\right)$ 和 $\left(X_2,Y_2\right)$ 两个消费组合中,消费者选择 $\left(X_1,Y_1\right)$ 而不是 $\left(X_2,Y_2\right)$,如果他是理性人,选择的目标是让自己的幸福感最大化,那么可以推断出比起 $\left(X_2,Y_2\right)$,消费者更偏好 $\left(X_1,Y_1\right)$。

假设消费者只消费 $X$ 和 $Y$ 两种商品,称 $\left(X_1,Y_1\right)$ 为一个\textbf{消费组合},它代表消费者对两种商品的消费量。在 $\left(X_1,Y_1\right)$ 和 $\left(X_2,Y_2\right)$ 两个消费组合中,消费者选择 $\left(X_1,Y_1\right)$ 而不是 $\left(X_2,Y_2\right)$,如果他是理性人,选择的目标是让自己的幸福感最大化,那么可以推断出比起 $\left(X_2,Y_2\right)$,消费者更偏好 $\left(X_1,Y_1\right)$。

假设消费者只消费 $X$ 和 $Y$ 两种商品,称 $\left(X_1,Y_1\right)$ 为一个\textbf{消费组合},它代表消费者对两种商品的消费量。在 $\left(X_1,Y_1\right)$ 和 $\left(X_2,Y_2\right)$ 两个消费组合中,消费者选择 $\left(X_1,Y_1\right)$ 而不是 $\left(X_2,Y_2\right)$,如果他是理性人,选择的目标是让自己的幸福感最大化,那么可以推断出比起 $\left(X_2,Y_2\right)$,消费者更偏好 $\left(X_1,Y_1\right)$。

假设消费者只消费 $X$ 和 $Y$ 两种商品,称 $\left(X_1,Y_1\right)$ 为一个\textbf{消费组合},它代表消费者对两种商品的消费量。在 $\left(X_1,Y_1\right)$ 和 $\left(X_2,Y_2\right)$ 两个消费组合中,消费者选择 $\left(X_1,Y_1\right)$ 而不是 $\left(X_2,Y_2\right)$,如果他是理性人,选择的目标是让自己的幸福感最大化,那么可以推断出比起 $\left(X_2,Y_2\right)$,消费者更偏好 $\left(X_1,Y_1\right)$。

假设消费者只消费 $X$ 和 $Y$ 两种商品,称 $\left(X_1,Y_1\right)$ 为一个\textbf{消费组合},它代表消费者对两种商品的消费量。在 $\left(X_1,Y_1\right)$ 和 $\left(X_2,Y_2\right)$ 两个消费组合中,消费者选择 $\left(X_1,Y_1\right)$ 而不是 $\left(X_2,Y_2\right)$,如果他是理性人,选择的目标是让自己的幸福感最大化,那么可以推断出比起 $\left(X_2,Y_2\right)$,消费者更偏好 $\left(X_1,Y_1\right)$。

假设消费者只消费 $X$ 和 $Y$ 两种商品,称 $\left(X_1,Y_1\right)$ 为一个\textbf{消费组合},它代表消费者对两种商品的消费量。在 $\left(X_1,Y_1\right)$ 和 $\left(X_2,Y_2\right)$ 两个消费组合中,消费者选择 $\left(X_1,Y_1\right)$ 而不是 $\left(X_2,Y_2\right)$,如果他是理性人,选择的目标是让自己的幸福感最大化,那么可以推断出比起 $\left(X_2,Y_2\right)$,消费者更偏好 $\left(X_1,Y_1\right)$。

假设消费者只消费 $X$ 和 $Y$ 两种商品,称 $\left(X_1,Y_1\right)$ 为一个\textbf{消费组合},它代表消费者对两种商品的消费量。在 $\left(X_1,Y_1\right)$ 和 $\left(X_2,Y_2\right)$ 两个消费组合中,消费者选择 $\left(X_1,Y_1\right)$ 而不是 $\left(X_2,Y_2\right)$,如果他是理性人,选择的目标是让自己的幸福感最大化,那么可以推断出比起 $\left(X_2,Y_2\right)$,消费者更偏好 $\left(X_1,Y_1\right)$。

假设消费者只消费 $X$ 和 $Y$ 两种商品,称 $\left(X_1,Y_1\right)$ 为一个\textbf{消费组合},它代表消费者对两种商品的消费量。在 $\left(X_1,Y_1\right)$ 和 $\left(X_2,Y_2\right)$ 两个消费组合中,消费者选择 $\left(X_1,Y_1\right)$ 而不是 $\left(X_2,Y_2\right)$,如果他是理性人,选择的目标是让自己的幸福感最大化,那么可以推断出比起 $\left(X_2,Y_2\right)$,消费者更偏好 $\left(X_1,Y_1\right)$。

偏好有“作比较”的含义,比较消费组合是消费者在对它们按照偏好程度排序。我们知道在实数比较大小中使用的符号是 $\ge$,\textbf{偏好关系}的符号则是 $\succsim$,不同符号突出了它们是不同的序关系。$\succsim$ 由 $\succ$ 和 $\sim$ 组成,分别指的\textbf{严格偏好}和\textbf{无差异},合起来 $\succsim$ 代表“至少一样好”。

假设小明很喜欢吃苹果和香蕉,有多少吃多少,那么一定有 $(\text{苹果},\text{香蕉})=\left(3,4\right)\succsim\left(2,1\right)$,因为前一个组合对两种水果的消费量都大于后一个组合。但我们会发现 $\left(3,4\right)$ 和 $\left(4,3\right)$ 这样两个组合我们无法判断小明会选择哪一个,选择的结果只有小明自己知道。说明偏好关系确实能代表消费者的喜好,但外人不总能判断出消费者的偏好\footnote{数学中把这样的序关系称作偏序关系,其实 $\succ$ 这样弯曲的大于号正是偏序关系的符号}。所以我们要引入接下来的内容,把研究中心放在我们能判断的商品组合中。

\end{document}
\input{13-ProducerTheory}
\input{14-MarketTheory}
\input{15-WelfareTheory}
\input{16-FactorTheory}
\input{17-GameTheory}
 
\input{21-EconomicData}
\input{22-LongTermEconomics}
 
\input{24-ShortEconomics}
\input{25-OpenEconomics}
\input{26-MacroPolicy}
\input{27-MicroFoundation}
\input{28-SchoolnEvent}

% 增加段间距
\setlength{\parskip}{0\baselineskip}


% 开始附录
\backmatter
%\setboolean{bookmarksnumbered}{false}
%\titlecontents{chapter}[0em]{\bf}{}{}{\titlerule*[]{}\contentspage}

\balance
\printindex
\addcontentsline{toc}{chapter}{索引}

\clearpage
\end{spacing}
\end{document}